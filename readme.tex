\documentclass[12pt]{article}
\usepackage{lingmacros}
\usepackage{tree-dvips}
\usepackage{xcolor}
\usepackage{listings}
\usepackage{enumitem}
\setlength{\parindent}{0em}
\begin{document}

%%%%%%%%%%%%%%%%%%%%%%%%%%%%%%%%%%%%%%%%%%%%%%%%%%%%%%%%%%%%%%%%%

\section*{mdeshell}

mdeshell is a bash shell based modeling environment that facilitates the definition of XML-based EMF metamodels/models and the execution of Model-2-Text transformation based on Model-Driven Engineering system implemented by the Eclipse Emfatic and the Epsilon Flexmi/EGL/EGX.  \\

Project website: https://github.com/vorachet/mdeshell \\

Maintainer: Vorachet Jaroensawas $<$vorachet@gmail.com$>$ \\

Limitation: mdeshell is tested on macOS 11.1 (Java 13) and Ubuntu 18.04 (Java 11). mdeshell does not provide an environment for MS Windows users. 


%%%%%%%%%%%%%%%%%%%%%%%%%%%%%%%%%%%%%%%%%%%%%%%%%%%%%%%%%%%%%%%%%

\section*{Getting started}
\noindent\rule{8cm}{0.4pt}
\begin{lstlisting}
  $ git clone https://github.com/vorachet/mdeshell.git
  $ cd mdeshell
\end{lstlisting}
\noindent\rule{8cm}{0.4pt}

%%%%%%%%%%%%%%%%%%%%%%%%%%%%%%%%%%%%%%%%%%%%%%%%%%%%%%%%%%%%%%%%%

\section*{Running demo projects}

Note that you will need a Java Runtime for executing a mdeshell shell script 

\subsection*{01\_getting\_started - A simplfied C program}
\noindent\rule{8cm}{0.4pt}
\begin{lstlisting}
  $ cd mdeshell
  $ ./run.sh 

  Choose project number:
  1) 01_getting_started
  2) 02_my_shell_program
  #?  <------ let enter 1

  Project: 01_getting_started 

  Generating...

  Done! Note that location of generated files 
  will be specified by your .egx files
\end{lstlisting}
\noindent\rule{8cm}{0.4pt}

This demo helps check the needed software runtime on your computer. By checking the expected outputs, myproject.cpp and myproject.txt should be generated

\noindent\rule{8cm}{0.4pt}
\begin{lstlisting}
  The expected output files

  mdeshell
    /projects
      /01_getting_started
        /generated/
          myproject.cpp
          myproject.txt
\end{lstlisting}
\noindent\rule{8cm}{0.4pt}

\subsection*{02\_my\_shell\_program - Interactive shell program}

This demo demonstrates the basic application of the Epsilon EGL/EGX language. You will practice to create your own shell script that comes with interactive menus.

\noindent\rule{8cm}{0.4pt}
\begin{lstlisting}
  $ cd mdeshell
  $ ./run.sh 

  Choose project number:
  1) 01_getting_started
  2) 02_my_shell_program
  3) dashboard
  #? <------ let enter 1

  Project: 02_my_shell_program

  Generating...

  Done! Note that location of generated files
  will be specified by your .egx files

  The expected output files

  mdeshell
    /projects
      /02_my_shell_program
        /generated/
          MyUnixJobs.sh

  Test the script MyUnixJobs.sh

  $ sh MyUnixJobs.sh 
    1) showMyComputerName     3) findFiles
    2) showNetworkInterfaces  4) quit

\end{lstlisting}
\noindent\rule{8cm}{0.4pt}

%%%%%%%%%%%%%%%%%%%%%%%%%%%%%%%%%%%%%%%%%%%%%%%%%%%%%%%%%%%%%%%%%

\pagebreak 
\section*{Usage Guide}

\subsection*{System files and directories}

\noindent\rule{8cm}{0.4pt}
\begin{lstlisting}
  mdeshell
    /libs/*                (1)
    /projects/*            (2)
    run.sh                 (3)
\end{lstlisting}
\noindent\rule{8cm}{0.4pt}

\begin{itemize}[label={}]
\itemsep0em
  \item (1) Java libraries required by mdeshell. (2) Root directory for your projects. (3) A mdeshell script that provides interactive command line for mdeshell workflow
\end{itemize} 

\subsection*{Organization of user project files}

\noindent\rule{8cm}{0.4pt}
\begin{lstlisting}
  mdeshell
    /projects
      /{YoutProjectName}  (1) USER-DEFINED NAME
        /inputs           (2) FIXED NAME
          metamodel.emf   (3) FIXED NAME
          *.flexmi        (4) USER-DEFINED NAME
          *.egl           (5) USER-DEFINED NAME
          *.egx           (6) USER-DEFINED NAME
\end{lstlisting}
\noindent\rule{8cm}{0.4pt}

\begin{itemize}[label={}]
\itemsep0em
  \item (1) Directory name of your project. (2) A fixed directory name ("inputs") containing user modeling files. (3) A fixed file name metamodel ("metamodel.emf") described by the Emfatic language. (4) One or more models (*.flexmi) described by the Epsilon Flexmi language. (5) One or more generation templates (*.egl) described by the Epsilon EGL language. (6) One or more generation task specifications (*.egx) described by the Epsilon EGX language
\end{itemize} 


%%%%%%%%%%%%%%%%%%%%%%%%%%%%%%%%%%%%%%%%%%%%%%%%%%%%%%%%%%%%%%%%%

\pagebreak 
\section*{The open source libraries mdeshell depends on}

This section gives thanks to the open-source projects mdeshell depends on. 

\subsubsection*{Emfatic - https://www.eclipse.org/emfatic }

Emfatic is a textual syntax for EMF Ecore metamodels.

\subsubsection*{Epsilon - https://www.eclipse.org/epsilon}

Epsilon is a family of Java-based scripting languages for automating common model-based software engineering tasks, such as code generation, model-to-model transformation, and model validation, that work out of the box with EMF (including Xtext and Sirius), UML, Simulink, XML and other types of models. Epsilon also includes Eclipse-based editors and debuggers, convenient reflective tools for textual modeling and model visualization, and Apache Ant tasks.

\begin{itemize}[label={}]
\itemsep0em
  \item (1) Flexmi - https://www.eclipse.org/epsilon/doc/flexmi/ (2) The Epsilon Generation Language (EGL) - https://www.eclipse.org/epsilon/doc/egl/ 
  \item (3) The Epsilon EGL Co-Ordination Language (EGX)
  \item - https://www.eclipse.org/epsilon/doc/egx/ 
\end{itemize} 




\subsubsection*{The open-source Java libraries and free Java libraries shared by this project}
\begin{lstlisting}
  epsilon-1.5.1-kitchensink.jar, guava-23.0.jar,
  org.eclipse.core.resources_3.13.600.v20191122-2104.jar,
  org.eclipse.core.runtime-4.3.1.jar,
  org.eclipse.emf.emfatic.core_0.8.0.201507261242.jar,
  org.eclipse.equinox.common_3.10.600.v20191004-1420.jar,
  org.eclipse.equinox.registry_3.8.600.v20191017-2055.jar,
  org.eclipse.gymnast.runtime.core_0.8.0.201507261242.jar
  EGXRunner.jar, Emfatic2Ecore.jar, Flexmi2Xmi.jar
\end{lstlisting}

\end{document}
